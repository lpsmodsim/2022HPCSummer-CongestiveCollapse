\documentclass{article}

%graphics
\usepackage{graphicx}
\graphicspath{{./images/}}

\usepackage{float}

% margins of 1 inch:
\setlength{\topmargin}{-.5in}
\setlength{\textheight}{9.5in}
\setlength{\oddsidemargin}{0in}
\setlength{\textwidth}{6.5in}

\usepackage{hyperref}
\hypersetup{
    colorlinks=true,
    linkcolor=blue,
    filecolor=magenta,      
    urlcolor=cyan,
    pdftitle={Overleaf Example},
    pdfpagemode=FullScreen,
    }

% Relevant links and notes
%
%	
%	https://en.wikipedia.org/wiki/Discrete-event_simulation
%	http://sst-simulator.org/
%	https://en.wikipedia.org/wiki/Network_congestion#Congestive_collapse
%	https://www.javatpoint.com/tcp-retransmission
%
%
%

\begin{document}

    % https://stackoverflow.com/a/3408428/1164295
    \begin{minipage}[h]{\textwidth}
        \title{2022 Future Computing Summer Internship Project:\\Modeling Congestive Collapse in a Discrete-Event Simulator to Measure its Existence in a Simulation.}
        \author{Nicholas Schantz}
        \date{\today}
            \maketitle
        \begin{abstract}
            \href{https://en.wikipedia.org/wiki/Network_congestion#Congestive_collapse}{Congestive collapse} is an issue involving \href{https://en.wikipedia.org/wiki/Reliability_(computer_networking}{reliable network protocols} where a large amount of retransmitted data clogs a network. This causes the network's useful throughput to decline. This research addresses the question as to if metrics exist to determine if this problem occurred in a simulated network. These metrics are useful for network architects who are unaware of this problem, because they can better understand how to avoid causing this problem in network simulations. The \href{https://en.wikipedia.org/wiki/Discrete-event_simulation}{discrete-event simulator} (DES) framework called \href{http://sst-simulator.org/}{Structural Simulation Toolkit} (SST) is used to simulate this activity and find metrics to determine if it exist in a simulation. A model is created of a simple reliable network in which multiple nodes send packets to one receiving node. The receiving node is flooded with retransmitted data over time to create congestive collapse in the network. Data from the receiving node is analyzed and results reveal that a receiver's queue depth and ratio of first packets to duplicate packets directly measure the existence of congestive collapse. The ratio of \href{https://en.wikipedia.org/wiki/Goodput}{goodput} to total \href{https://en.wikipedia.org/wiki/Network_throughput}{throughput} can also indirectly measure congestive collapse.
        \end{abstract}
    \end{minipage}

\ \\
% see https://en.wikipedia.org/wiki/George_H._Heilmeier#Heilmeier's_Catechism

%\maketitle

\section{Project Description} % what problem is being addressed? 

The challenge addressed by this work is to map the networking problem congestive collapse to a discrete event simulator. \href{https://www.sandia.gov/}{Sandia National Laboratories'} Structural Simulation Toolkit is a discrete-event simulator framework which is used to simulate the problem. The problem is simulated to identify mathematical and logical conditions that cause the problem. This information is used to develop metrics to identify that the problem has occurred in simulated distributed systems.

\section{Motivation} % Why does this work matter? Who cares? If you're successful, what difference does it make?

Identifying the metrics for detecting congestive collapse will be vital for developing distributed systems that can avoid congestive collapse from occurring during communication. Network architects may be unaware of this problem when they begin to model and simulate a network. The metrics found in this research can help them catch these problems when they occur during the the simulations development. The SST models created will demonstrate how the problem can occur in a simulation. These results allow for the problem to be avoided during simulation so it does not occur when the system is put into production.

A second outcome of this work is to develop examples and documentation of SST simulations that new users can utilize. Currently, SST is primarily used for \href{https://en.wikipedia.org/wiki/High-performance_computing}{High-Performance Computing} (HPC) simulations and the majority of SST examples revolve around modeling HPC systems. However, the SST framework utilizes a powerful discrete-event simulator that can model more than just HPC systems. The models and documentation created for this project show off the discrete-event simulator for simulations not relating to HPC systems.

\section{Prior work} % what does this build on?
TODO

\section{Running the Model}

The software developed to respond to this challenge can be run on one laptop.

\noindent The software is available on \href{https://github.com/lpsmodsim/2022HPCSummer-CongestiveCollapse}{https://github.com/lpsmodsim/2022HPCSummer-CongestiveCollapse}.\newline

\noindent To run the software:
\noindent TODO: Add prerequisites (get SST singularity container, add additions.sif, etc))

\begin{verbatim}
	git clone https://github.com/lpsmodsim/2022HPCSummer-CongestiveCollapse
	cd 2022HPCSummer-CongestiveCollapse
	make
\end{verbatim}

\noindent To re-run the software:

\begin{verbatim}
	make clean
	make
\end{verbatim}

\noindent The simulation can be modified by editing the \href{http://sst-simulator.org/SSTPages/SSTUserPythonFileFormat/}{python driver file}, it is located at:

\begin{verbatim}
2022HPCSummer-CongestiveCollapse/tests/congestionCollapse.py
\end{verbatim}

\noindent TODO: Include example output\newline

\section{The Model}

The model is made up of sender components and receiver components who send and receive packets respectively. There is \textit{n} senders connected to one receiver. The senders will send a set number of packets to the receiver and the receiver will enqueue the packets to an infinite queue. The receiver will process a set number of packets and respond to the senders with acknowledgments for the packets it has processed. The sender keeps track of every packet sent and if it does not receive an acknowledgment for that packet in a set time, it will retransmit the packet.

\begin{figure}[H]
	\includegraphics[scale=0.5]{model}
	\centering
	\caption{Example of model with one sender and receiver node}
\end{figure}

Key components that are logged by the receiver component are as follows:

\begin{itemize}
	\item Throughput - the number of new and retransmitted packets that are processed by the receiver per second.
	\item Goodput - the number of new packets that are processed by the receiver per second.
	\item Queue Depth - The number of packets in the queue.
	\item Queue Entries - Number of new and retransmitted packets entering the queue per second.
\end{itemize}


\section{Result} % conclusion/summary

To determine if congestive collapse exist in a network. Three candidate metrics were determined:

\begin{itemize}
	\item Ratio of new packets to retransmitted packets in a receiver's queue.
	\item Receiver's queue depth.
	\item Ratio of goodput and throughput of the receiver.
\end{itemize}

The ratio of new packets and total packets in the receiver's queue will give an idea if more retransmitted packets are being enqueued by the receiver if the ratio decreases over time. However, only using this metric could produce false positives for detecting congestive collapse. For example, during a situation where there is low network traffic and the receiver's queue is close to empty, if one sender happens to retransmit packets that were previously lost, this will decrease the measurement even though congestive collapse did not occur.
To resolve this issue, queue depth will also be measured to determine if there is a low ratio of new packets and total packets while the receiver's queue depth remains consistently high. In this case, a network with a high queue depth is clogged and if its ratio of packets insist of mostly retransmitted packets, congestive collapse exist in the network.

%Calculating a ratio between the useful throughput and the total throughput of a receiver gives insight to how many duplicate packets have been enqueued in a receiver. However, if this ratio is low due to corrupted packets, that does not indicate congestive collapse. In this case queue depth is measured and if it is largely being filled with duplicate packets rather than new packets, then we can declare that the network is being flooded with retransmissions. In this case these metrics can identify that congestive collapse exist in the network.

An example of the problem is simulated and gathered data is shown below. In this SST simulation, a sender component is sending packets at three times the rate that a receiver component can process them at. The receiver's queue fills quickly and the receiver is unable to send acknowledgments under the sender's timeout time. Retransmissions are sent which cause more packets to fill the queue and the ratio of new packets to total packets in the queue exponentially decays to zero.

\begin{figure}[H]
	\includegraphics[scale=0.5]{newpackets}
	\centering
	\caption{Amount of new packets entering the queue each second for one hundred second}
\end{figure}

\begin{figure}[H]
	\includegraphics[scale=0.5]{queue}
	\centering
	\caption{Infinite queue which increases over time.}
\end{figure}

As the queue grows, starting at around 15 seconds, the ratio of new and retransmitted packets in the queue begins to exponentially decay. In a short span of time, only retransmitted packets are being sent to the receiver since the receiver cannot process the packets in its queue.

\begin{figure}[H]
	\includegraphics[scale=0.5]{throughput}
	\centering
	\caption{Ratio of goodput and total throughput over one hundred seconds}
\end{figure}

\section{Future Work}
TODO

\end{document}
